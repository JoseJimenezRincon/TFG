\chapter{Conclusiones}\label{chap:conclusiones}

\drop{E}n el trabajo realizado a lo largo de este \ac{TFG} se ha cumplido el objetivo principal del mismo que consistía en conseguir que un vehículo llegue desde un punto de origen a un punto de destino esquivando los obstáculos que aparezcan en el escenario en tiempo real, mediante el uso de hardware reconfigurable. 

Como consecuencia del desarrollo de este proyecto se ha llegado a la conclusión de que el uso de hardware reconfigurable ha sido un acierto en cuanto a mejora de rendimiento. Ya que gracias a su uso se ha mejorado $654,9$  veces el tiempo de ejecución en software de la aplicación.

Para la realización de este \ac{TFG} ha sifo necesario integrar los conocimientos recibidos de las siguientes asignaturas de la carrera: Diseño de Sistemas Basados en Microprocesador, Sistemas Inteligentes, Ingeniería del Software, Redes, Sistemas Operativos, Gestión y Administración de Redes entre otras.

\section{Objetivos logrados}

Se han logrado cumplir todos los objetivos específicos relacionados con \nameref{sec:DetecciónObstáculos}; \nameref{sec:DetecciónVehiculo}; \nameref{sec:CalculoTrayectoria}; \nameref{sec:GenerarMovimientos}; \nameref{sec:ComunicaciónVehículo}; \nameref{sec:MovimientoVehículo};  \nameref{sec:TiempoReal}.

\section{Trabajo futuro}

Ahora bien, gracias al trabajo realizado en este proyecto se ha podido pensar en varias optimizaciones que se pueden realizar en el sistema para mejora su funcionamiento. Estas son algunas de ellas:

\begin{enumerate}
\item Mejorar la detección de obstáculos. La resta de imágenes funciona bien para la mayoría de casos. Pero el umbral con el cual se elimina el ruido que se ha generado en la toma de las fotografías \emph{A} y \emph{B} no es dinámico. Es decir, es necesario realizar una configuración para cada escenario en el cual se instala el sistema. Este hecho limita mucho la compatibilidad entre configuraciones de distintos escenario. Se propone añadir un algoritmo que calcule la configuración en función de la altura y la luminosidad ya que si para el mismo escenario se sitúa la cámara a una altura distinta el umbral que se había configurado para el escenario probablemente no de unos buenos resultados para todos los casos. Al igual que el hecho de ejecutar la aplicación en días distintos en los que el tiempo cambie puede dar lugar a que el umbral no funcione como se espera debido al cambio de luz del entorno.
\item Mejorar la aplicación del algoritmo de segmentación de la imagen en macro-bloques. El algoritmo funciona correctamente en cuanto a calcular el número de objetos que aparecen en un bloque. El problema es que se ha simplificado el algoritmo debido a que se pueden dar situaciones que den lugar a una colisión. Es decir, el pre-procesamiento que se realiza del mapa en resolución \emph{FullHD} simplifica el resultado de tal manera que si el número de objetos por bloque es mayor que uno, pone ese bloque como ocupado. Aunque el bloque esté mayormente vacío. Como consecuencia la ruta que calcula posteriormente el algoritmo A* no es óptima en cuanto a distancia aunque la naturaleza del algoritmo sea calcular la ruta más corta posible. Esto es debido a que el hecho de simplificar los 900 valores posibles ($30x30$) a 2 (Ocupado o libre) evita que se puedan visitar nodos que son una mejor opción en cuanto a distancia más corta. Por ello, como mejora se invita a analizar si los bloques que no están ocupados completamente son transitables. Ya que se puede dar la circunstancia en la que una pequeña línea atraviese el bloque de tal forma que no se pueda pasar por el, aunque el bloque esté prácticamente libre de objetos. Si se consigue analizar si un bloque es transitable se puede dejar el valor calculado previamente por el algoritmo y así se mejorará la posterior aplicación del algoritmo A* que calculará una ruta más corta para el mismo escenario.
\item Implementar todo el procesamiento de imágenes en la FPGA de la placa \emph{Zedboard}. En este \ac{TFG} solo se ha implementado en \ac{FPGA} el algoritmo de cálculo de trayectoria. Si se consiguiera utilizar hardware reconfigurable para todos los algoritmos del proyecto se podría mejorar el tiempo de ejecución del sistema, permitiendo así poder aumentar la velocidad de movimiento del vehículo a causa de su mejor capacidad de reacción frente a cambios en el escenario.
\end{enumerate}